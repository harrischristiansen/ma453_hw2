\documentclass[11pt]{article}
\input{headers2}

\usepackage{fancyhdr}   
\pagestyle{fancy}      
\lhead{MA453 Spring 2018 - Homework 2}               
\rhead{Harris Christiansen (christih@purdue.edu)}

\usepackage{mathrsfs}
\usepackage[strict]{changepage}  
\newcommand{\nextoddpage}{\checkoddpage\ifoddpage{\ \newpage\ \newpage}\else{\ \newpage}\fi}

\begin{document}

\title{Homework 2}
\date{p39 B.5 and B.6, p40 C.7, p48 A.1, p49 B.4, p50 D.1 and D.3}
\maketitle

\thispagestyle{fancy}  
\pagestyle{fancy}      

\begin{enumerate}

%%% Problem p39 B.5
\item {\bfseries p39 B.5.} Prove the following is true in every group $G$, or give a counterexample. \\
  For every $x \in G$, there is some $y \in G$ such that $x = y^2$. (This is the same as saying that every element of G has a "square root.")
  
  {\bfseries Solution.}
  
  False. There are many groups where this is not true. \\
  
  Consider the Klein four-group: \\
  \begin{tabular}{l | l l c r}
	* & e & a & b & c \\
	\hline
	e & e & a & b & c \\
	a & a & e & c & b \\
	b & b & c & e & a \\
	c & c & b & a & e \\
  \end{tabular}
  
  In the Klein four-group, for any $y \in G$, $y^2 = e$. Thus, this is not true. \qed \\

 
%%% Problem p39 B.6
\item {\bfseries p39 B.6.} Prove the following is true in every group $G$, or give a counterexample. \\
  For any two elements $x$ and $y$ in $G$, there is an element $z$ in $G$ such that $y = xz$
  
  {\bfseries Solution.}
  
  Let $x, y \in G$. Assume $y = xz$ \\
  $x^{-1}y = x^{-1}xz$ \\
  $x^{-1}y = ez$ \\
  $x^{-1}y = z$
  
  Because $x \in G$, we know $x^{-1} \in G$. Since the group $G$ is closed under the group operation, we know $x^{-1}y \in G$.
  
  Thus we know there exists $z \in G$, where $z = x^{-1}y$, such that $y = xz$. \qed \\

%%% Problem p40 C.7
\item {\bfseries p40 C.7.} Assuming that $a$ and $b$ commute, prove the following: \\
  $ab = ba$ iff $aba^{-1}b^{-1} = e$
  
  {\bfseries Solution.}
  
  \begin{itemize}
  
	  \item Assume $ab = ba$, show $aba^{-1}b^{-1} = e$: \\
	  $(ab)a^{-1} = (ba)a^{-1}$ \\
	  $aba^{-1} = be = b$ \\
	  $(aba^{-1})b^{-1} = bb^{-1}$ \\
	  $aba^{-1}b^{-1} = e$ \qed \\
	  
	  
	  \item Assume $aba^{-1}b^{-1} = e$, show $ab = ba$ \\
	  $(aba^{-1}b^{-1})b = eb$ \\
	  $aba^{-1}e = eb$ \\
	  $aba^{-1} = b$ \\
	  $aba^{-1}a = ba$ \\
	  $abe = ab = ba$ \qed \\
	  
	  \item Thus, $ab = ba$ iff $aba^{-1}b^{-1} = e$ \qed
  
  \end{itemize}
  
\newpage
 
%%% Problem p48 A.1
\item {\bfseries p48 A.1.} Determine whether or not $H$ is a subgroup of $G$:
  
  $G = \langle \mathbb{R},+\rangle, H = \{\log a : a \in \mathbb{Q}, a>0\}$
  
  {\bfseries Solution.}
  
  $H$ {\bfseries is} a subgroup of $G$, because given $x \in H \rightarrow x^{-1} \in H$ and given $x,y \in H \rightarrow xy \in H$. \\

%%% Problem p49 B.4
\item {\bfseries p49 B.4.} Show that $H$ is a subgroup of $G$:
  
  $G = \langle \mathscr{C}(\mathbb{R}),+\rangle, H = \{f \in \mathscr{C}(\mathbb{R}) : \int_{0}^{1} f(x) dx = 0\}$
  
  {\bfseries Solution.}
  
  \begin{itemize}
  
	  \item Given $x \in H$, show $x^{-1} \in H$: \\
	  $x \in \mathscr{C}(\mathbb{R})$
	  $\int_{0}^{1} f(x) dx = 0$ \\
	  Since the $x^{-1}$ still satisfies an integral of 0, we know $x^{-1} \in H$. \qed \\
	  
	  \item Given $x,y \in H$ show $xy \in H$: \\
	  $x,y \in \mathscr{C}(\mathbb{R})$ \\
	  $\int_{0}^{1} f(x) dx = 0$ \\
	  $\int_{0}^{1} f(y) dy = 0$ \\
	  $\int_{0}^{1} f(x) dx + \int_{0}^{1} f(y) dy = 0$ \\
	  Thus, since adding the two integrals still produces 0, the result is also in H. $xy \in H$ \qed \\
	  
	  \item Since we have shown $x^{-1} \in H$ and $xy \in H$, we can conclude that $H$ is a subgroup of $G$. \qed
  
  \end{itemize}
 
\newpage
 
%%% Problem p50 D.1
\item {\bfseries p50 D.1.} Let $G$ be a group
  
  If $H$ and $K$ are subgroups of a group $G$, prove that $H \cap K$ is a subgroup of $G$. (Remember that $x \in H \cap K$ iff $x \in H$ and $x \in K$.)
  
  {\bfseries Solution.} To prove $H \cap K$ is a subgroup of G, we must show: \\
  if $f \in S$ then $f^{-1} \in S$ \\
  if $f,g \in S$ then $fg \in S$ \\
  
  Let $H, K$ be subgroups of $G$. \\
  We know by definition that for every $x \in H \cap K$, $x \in H$ and $x \in K$ \\
  Since $H$ and $K$ are subgroups, we also know $x^{-1} \in H$ and $x^{-1} \in K$, thus $x^{-1} \in H \cap K$ \qed \\
  
  For $x,y \in H \cap K$, we know $x,y \in H$ and $x,y \in K$ \\
  Since $H$ and $K$ are subgroups, they are closed under their operation. Thus $xy \in H$ and $xy \in K$ \\
  Thus $xy \in H \cap K$. \qed \\
  
  Since we have shown $x^{-1} \in H \cap K$ and $xy \in H \cap K$, we can conclude that $H \cap K$ is a subgroup of $G$. \qed \\

%%% Problem p50 D.3
\item {\bfseries p50 D.3.} Let $G$ be a group
  
  By the \textit{center} of a group $G$ we mean the set of all elements of $G$ which commute with every element of $G$, that is,
  
  $C = \{ a \in G : ax = xa$ for every $x \in G\}$
  
  Prove that $C$ is a subgroup of $G$.
  
  {\bfseries Solution.}
  
  \begin{itemize}
  
	  \item Given $f \in C$, show $f^{-1} \in C$: \\
	  Given $f \in C, a \in G$, by definition of \textit{center}, $af = fa \; \forall \; f \in G$ \\
	  $f^{-1}af = f^{-1}fa$ \\
	  $f^{-1}af = ea = a$ \\
	  $f^{-1}aff^{-1} = af^{-1}$ \\
	  $f^{-1}ae = f^{-1}a = af^{-1}$ \\
	  Thus, by definition of $C$, $f^{-1} \in C$ \qed \\
	  
	  \item Given $f,g \in C, a \in G$ show $fg \in C$: \\
	  We know $f(ga) = (ga)f$ since $ga \in G$ and $f \in C$ \\
	  $f(ga) = g(af)$, by associativity \\
	  $f(ga) = (af)g$, since $g \in C$ and $af \in G$ \\
	  $f(ga) = afg$, thus by definition of $C$, $fg \in C$ \qed \\
	  
	  \item Since we have shown $f^{-1} \in C$ and $fg \in C$, we can conclude that $C$ is a subgroup of $G$. \qed
  
  \end{itemize}


\end{enumerate}

\end{document}
